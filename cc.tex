\documentclass[a4paper,10pt]{article}
\usepackage{amsmath}
\usepackage[utf8]{inputenc}
\usepackage{amssymb}
%opening
\title{3x+1 Problem}
\author{Joshua Wong}

\begin{document}

\maketitle
  The Collatz Problem, also known as the \(3x+1\) problem, is defined as:
  \[
    c(x) = \left.
    \begin{cases}
        x/2, & \text{if } x = 0 {\pmod {2}}\\
        3x+1, & \text{if } x = 1 {\pmod {2}} \\
    \end{cases}
    \right\}, x \in \mathbb{N{^+}}
  \]
  Or the "shortcut" version:
  \[
    t(x) = \left.
    \begin{cases}
        x/2, & \text{if } x = 0 {\pmod {2}}\\
        (3x+1)/2, & \text{if } x = 1 {\pmod {2}} \\
    \end{cases}
    \right\}, x \in \mathbb{N{^+}}
  \]
  \[
    f(x) = \left.
    \begin{cases}
        x/4, & \text{if } x = 0 {\pmod {4}}\\
        (3x+1)/2, & \text{if } x = 2 {\pmod {4}} \\
        (3x+1)/2, & \text{if } x = 1 \text{ or } 3 {\pmod {4}} \\
    \end{cases}
    \right\}, x \in \mathbb{N{^+}}
  \]
  This process will eventually reach the number 1, regardless of which positive integer is chosen initially.
  \section{Proving odd inputs of Problem are even}
  The second part of the Collatz Conjecture (3x+1) applies to every number in the set: \(\{ x \in \mathbb{N{^+}} \ | \ 2x+1 \}\). \\
  Let c be a random number from this set. \(c(c) = 3(2x+1)+1 \) \\
  Factor out the variables: \(6x+3+1 = 6x+4 = 2(3x+2)\) \\
  Since the expression does not follow the form for an odd number, \(2x+1\), it is an even number.
  Therefore, all odd inputs of the Collatz Conjecture are even.
  \section{Proving the Collatz Problem for the set $2^{n}$}
  \(\{x \in \mathbb{N{^+}}: 2{^x}\}\) \\
  In this case, \(c^{(log_2{x})}(x) = 1\). \\
  Ex. \(x = 2, c(2) = \frac{2}{2} = 1\) \\
  Therefore, the conjecture for the set $2^{n}$ is true.
  \section{Finding functions $c{^2}(x)$ and beyond}
  Since $c(x)$ is a recursive function, we can repersent it as $c(c(x))$ or $c{^2}(x)$ and beyond.
  Let's start with the case of \(\{x \in \mathbb{N{^+}} : 2x+1\}\) \\
  \(c^2(x) = \frac{3x+1}{2}\) \\
  Using the proof discussed earlier, as odd inputs of \(c(x)\) always return a even value, we can deduce that \(c^2(x) = \frac{3x+1}{2}, x \in \mathbb{N{^+}} \ | \ 2x+1\).
  \section{Finding the worst case number for $3x+1$}
  Given a number \(\{x \in \mathbb{N{^+}}: 2x + 1\}\) after \(t(x) = \frac{3x+1}{2}\) will still be part of this set, over time, the number will look like this (change $r$ to the amount of times you should run the Collatz shortcut function).
  \[
    w^{r}(x) = \frac{3x+1}{2}+\sum_{i=1}^{r-1} \frac{3^{i}x+3^{i}}{2^{(i+1)}}
  \]
  Or:
  \[
    w^{r}(x) = \left(\frac{3}{2}\right)^{r}*(x+1)-1
  \]
  Example: 31
  \[
     w^{5}(31) = \frac{3x+1}{2}+\sum_{i=1}^{5-1} \frac{3^{i}x+3^{i}}{2^{(i+1)}} = 242
  \]
  Since 31 leads to an even number, it will not break the Collatz Conjecture.
  \section{Finding the average case number for $3x+1$}
  Given a number \(\{x \in \mathbb{N{^+}}: 2x + 1\}\)
  \section{Finding the prime factors  for $3x+1=2^n$}
  Is there a pattern when it comes to prime factors of numbers of this format?
  \(\{x \in N{^+} : \frac{2^{2n}-1}{3}\}\)
  \section{Finding the zero series for $3x+1$}
  Find a mix of $c(x) = x/2, x \in N^{+} : 2x$ and $c(x) = 3x+1, x \in N^{+} : 2x + 1$ such that
  \[o^{r}(x) = \sum_{i=1}^{r-1} a_i = 0\]
  \[w^{r}(x) = \frac{3x+1}{2}+\sum_{i=1}^{r-1} \frac{3^{i}x+3^{i}}{2^{(i+1)}}\]
  \section{Finding the Limit of the Worst Case Function}
  \[
    \frac{3x+1}{2}+\sum_{i=1}^{\frac{\ln{\frac{2(3x+2)}{3(x+1)}}}{\ln{\frac{3}{2}}}} \frac{3^{i}x+3^{i}}{2^{(i+1)}}
  \]
\end{document}
