\documentclass[a4paper,10pt]{article}
\usepackage{amsmath}
\usepackage[utf8]{inputenc}
\usepackage{amssymb}

%opening
\title{3x+1 Problem}
\author{Joshua Wong}

\begin{document}

\maketitle
    The Collatz conjecture, also known as the \(3x+1\) problem is defined as:
  \[
    f(x) = \left.
    \begin{cases}
        x/2, & \text{if } x = 0 {\pmod {2}}\\
        3x+1, & \text{if } x = 1 {\pmod {2}} \\
        \end{cases}
    \right\}, x \in \mathbb{N{^+}}
  \]
  Or the "shortcut" version:
  \[
    f(x) = \left.
    \begin{cases}
        x/2, & \text{if } x = 0 {\pmod {2}}\\
        (3x+1)/2, & \text{if } x = 1 {\pmod {2}} \\
        \end{cases}
    \right\}, x \in \mathbb{N{^+}}
  \]
  This process will eventually reach the number 1, regardless of which positive integer is chosen initially. \\ \\
  %Let x be a
  %We can duduce that the Collatz conjecture will lead to
\end{document}
