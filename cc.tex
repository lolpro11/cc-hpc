\documentclass[a4paper,10pt]{article}
\usepackage{amsmath}
\usepackage[utf8]{inputenc}
\usepackage{amssymb}

%opening
\title{3x+1 Problem}
\author{Joshua Wong}

\begin{document}

\maketitle
  The Collatz conjecture, also known as the \(3x+1\) problem, is defined as:
  \[
    c(x) = \left.
    \begin{cases}
        x/2, & \text{if } x = 0 {\pmod {2}}\\
        3x+1, & \text{if } x = 1 {\pmod {2}} \\
        \end{cases}
    \right\}, x \in \mathbb{N{^+}}
  \]
  Or the "shortcut" version:
  \[
    s(x) = \left.
    \begin{cases}
        x/2, & \text{if } x = 0 {\pmod {2}}\\
        (3x+1)/2, & \text{if } x = 1 {\pmod {2}} \\
        \end{cases}
    \right\}, x \in \mathbb{N{^+}}
  \]
  This process will eventually reach the number 1, regardless of which positive integer is chosen initially.
  \section{Proving the Collatz Conjecture for the set $2^{n}$}
  \(\{x \in \mathbb{N{^+}}: 2{^x}\}\) \\
  In this case, \(c^{(log_2{x})}(x) = 1\). \\
  Ex. \(x = 2, c(2) = \frac{2}{2} = 1\) \\
  Therefore, the conjecture for the set $2^{n}$ is true.
  %$x * \prod\limits_{i=1}^{\log_2{x}} $
  %Let \(c\) be any number within \(\{\mathbb{N{^+}}\ | \ c = 0 \pmod {2} \}\). We put \(c\) into the \\ Collatz function \(c(c) = c/2 \). If \(c\) is even, then
  %We can duduce that the Collatz conjecture will lead to
\end{document}
